\documentclass{article}
\usepackage[margin=1.5cm,bottom=2cm]{geometry}
\usepackage{fancyhdr}
\usepackage{graphicx}
\usepackage[section]{placeins}
\pagestyle{fancy}

\title{Exam 1 Study Guide (Chapters 1-2)}
\date{}
\begin{document}
\maketitle

\section*{Chapter 1}
\begin{itemize}
	\item Know what thermal equilibrium means and how it relates to temperature
	\item The ideal gas law ($PV=NkT$) and how to use it \\
	Example Problems:
		\begin{itemize}
			\item 1.9, 1.10, 1.11, 1.16
		\end{itemize}
	\item Know how to use the equipartition theorem ($U=\frac{1}{2}NfkT$)\\
	Example Problems:
	\begin{itemize}
		\item 1.23,1.24
	\end{itemize}
	\item The first law: $\Delta U=Q+W$. Positive sign corresponds to energy entering the object.
	\item Be able to calculate $\Delta U$, $Q$ ,$W$, etc for compression/expansion of an ideal gas (both adiabatic and isothermal)\\
		Example Problems:
	\begin{itemize}
		\item 1.36,1.37
	\end{itemize}
	\item Heat capacities: definition ($C_V=\left(\frac{\partial U}{\partial T} \right)_V$, $C_P=\left(\frac{\partial U}{\partial T} \right)_P + P\left(\frac{\partial V}{\partial T}\right)_P$)
	\begin{itemize}
		\item Know how to measure capacities given $Q$ and $\Delta T$ and how to use the heat capacity to predict $\Delta T$ (1.41, 1.42)
	\end{itemize}
	
\end{itemize}
\section*{Chapter 2}
\begin{itemize}
	\item You should thoroughly understand the concept of a microstate and macrostate and what the difference is. (Would you be able to explain this concept to somebody else?)
	\item I don't care that you memorize the multiplicity of an Einstein solid, but be sure you know the context (what do $N$ and $q$ refer to?)\\
	Example Problems:
	\begin{itemize}
		\item 1.5-1.7
	\end{itemize}
	\item Be familiar with computing the microstates and macrostates of interacting systems (i.e. two interacting Einstein solids)
	\item Be able to describe the 2nd law in terms of macrostate multiplicity
	\item Know how to find the probability of a given macrostate ($\Omega_{state}/\Omega_{all}$)\\
		Example Problems:
	\begin{itemize}
		\item 1.8
	\end{itemize}
	\item Be familiar with the main results of section 2.4. You won't be asked to derive them, but you should know what they mean. This includes: the multiplicity of the Einstein solid in the high temperature limit, and the result that for very large $N$ ($N\sim10^{20}$) the multiplicity curve is so sharply peaked that any fluctuations within the peak are utterly unobservable (for all intents and purposes, a system at equilibrium is in the most likely macrostate)
	\item You won't be asked to derive the multiplicity of the mono-atomic ideal gas, but you should know what it means and how to use it (what is the relative probability of a macrostate with a given volume, energy, etc). Example: 1.27
	\item Definition of entropy: $S=k\ln\Omega$. Be able to find $\Delta S$ for common processes such as expansion, mixing, etc. Be able to express changes in entropy in terms of ratio of microstates before and after a process.
	\item Be familiar with the 2nd law in its most common form: $\Delta S_{total} \geq 0$
	\item Know what defines a reversible vs an irreversible process, and the relationship to entropy.
\end{itemize}
\end{document}