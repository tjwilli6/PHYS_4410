\question 

\begin{parts}
	\part[5] In terms of microstates and entropy, explain why a gas is almost certain to eventually fill the volume of its container (it will never spontaneously move to one side or occupy a smaller volume). \vspace{3cm}
	\part[15] Currently, our room is at a temperature of 300 K, pressure of 1 atm, and occupies a volume of 300 cubic meters. What is the probability that the gas in this room spontaneously contracts to 90\% of its original volume? (You should treat the gas as mono-atomic, even though it isn't).
\end{parts}