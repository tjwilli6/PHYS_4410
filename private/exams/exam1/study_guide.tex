\documentclass{article}
\usepackage[margin=1.5cm,bottom=2cm]{geometry}
\usepackage{fancyhdr}
\usepackage{graphicx}
\usepackage{amsmath}
\pagestyle{fancy}

\begin{document}
\fancyhead[L]{ \includegraphics[width=2cm]{au_logo.png} }
\fancyhead[R]{PHYS 2250: General Physics II}
\fancyfoot[C]{\thepage}
\vspace*{0cm}
\begin{center}
	{\LARGE \textbf{Exam I Study Guide}}\\
	\vspace{0.25cm}
	%{\Large Due: Friday, September 4}
\end{center}
\section*{Chapter 13}
%\subsection*{Core Concepts}
\begin{itemize}
	\item Definition of electric field:
		\begin{equation*}
		\vec{E}=\frac{\vec{F}}{q}
		\end{equation*}
		\begin{itemize}
			\item Know how to calculate electric field given force and charge
			\item Know how to calculate electric force given field and charge
			\item Be familiar with the direction of the force for both positive and negative particles. Positive charges experience a force in the same direction as $\vec{E}$. Negative charges experience a force in the opposite direction as $\vec{E}$.
			\item Example problems: P17, P20, P22
		\end{itemize}
	
	\item Electric field of a point charge
		\begin{equation*}
		\vec{E}=\frac{1}{4\pi\epsilon_0}\frac{q}{|\vec{r}|^2}\hat{r}
		\end{equation*}
		\begin{itemize}
			\item $\vec{r}$: the vector pointing from the charge to the point where the field is being measured ($\vec{r}_\mathrm{obs}-\vec{r}_\mathrm{src}$)
			\item Be able to sketch the field lines of a point charge (radially outward for a positive charge, radially inward for a negative charge)
			\item Example problems: P28, P33, P35, P36
		\end{itemize}
	
	\item The principle of superposition
		\begin{equation*}
		\vec{E}_\mathrm{total}=\vec{E}_1+\vec{E}_2+\vec{E}_3+...
		\end{equation*}
		\begin{itemize}
		\item The total electric field of several point charges is the vector sum of the individual fields of each charge
		\item Find the field vector of each charge, then add them together
		\item Make sure you know how to add vectors!
		\item Example problems: P47, P48, P49
		\end{itemize}

%\subsection*{Applications}

	\item Dipoles
	\begin{itemize}
		\item You will be given the expression for the electric field on-axis and perpendicular
		\begin{eqnarray*}
		|\vec{E}_\mathrm{dipole,on-axis}| \approx \frac{1}{4\pi\epsilon_0}\frac{2p}{r^3}\\
		|\vec{E}_\mathrm{dipole,perp}| \approx \frac{1}{4\pi\epsilon_0}\frac{p}{r^3}
		\end{eqnarray*}
		\item Remember how we derived these expressions. There is nothing magic about them, a dipole is just two point charges and we used superposition to find the field
		\item Know how to calculate the dipole moment $p=qs$
		\item Example problems: P52, P53, P57, P59
	\end{itemize}
\end{itemize}

\section*{Chapter 14}
%\subsection*{Core Concepts}
\begin{itemize}
	\item Be familiar with the behavior of charges on both insulators and conductors
	\item Be able to calculate the mutual force between a point charge and a neutral insulator
	\item Be familiar with the conditions for static equilibrium within a conductor
	\begin{itemize}
		\item $\bar{v}=0 \Leftrightarrow |\vec{E}_\mathrm{net}|=0$
		\item Only the total field $\vec{E}_\mathrm{net}$ needs to be 0 in equilibrium. There can still be a field due to an external charge, it is just exactly canceled by the induced field due to polarization inside the metal
	\end{itemize}
	\item What are the mechanisms for charging/discharging for both insulators and conductors?
	\begin{itemize}
		\item Insulators
		\begin{itemize}
			\item Charge by contact (excess charge stays put)
		\end{itemize}
		\item Conductors
			\item Charge/discharge by contact
			\item Charge by induction
			\item Grounding
	\end{itemize}
	\item Example problems: P29, P37, P45, P49, P50, P56, P61
\end{itemize}
\begin{table}[ht!]
	\renewcommand{\arraystretch}{2.5}
	\begin{tabular}{ccc}
		&\textbf{Insulators}&\textbf{Conductors}\\
		\textbf{Mobile Charges?}&No&Yes\\
		\textbf{Location of excess charge}&Anywhere&Surface only\\
		\textbf{Does excess charge spread?} & No & Yes, uniformly around the surface\\
		\textbf{$\vec{E}_\mathrm{net}$ inside} & Can be non-zero & 0 in equilibrium\\
		\textbf{Polarization}&Induced dipoles ($p=\alpha|\vec{E}|$)&Moving charges ($\bar{v}=u|\vec{E}|$)
	\end{tabular}
\end{table}

\section*{Chapter 15}
\begin{itemize}
	\item Be able to set up integral expressions for the electric field of uniform, one-dimensional charge distributions
	\begin{itemize}
		\item Find charge density (total charge $\div$ length).
		\item Divide the distribution into very small pieces.
		\item Find the charge $dq$ of a single piece (in terms of the charge density and a differential variable).
		\item Find $\vec{r}=\vec{r}_\mathrm{obs}-\vec{r}_\mathrm{src}$ for this piece, and use $dq$ and $\vec{r}$ to find the electric field $\vec{dE}$ of the piece of charge, assuming it is a point charge.
		\item The total field of the distribution is simply the integral of the field of every little piece $\vec{dE}$ over every possible charge location $\vec{r}_\mathrm{src}$. Find the bounds of integration, and write this expression in integral form.
		\item Be able to derive the field of a charged rod or ring
		\item Example problems: P27, P29, P30 (a)
	\end{itemize}
	\item Be able to use superposition to find the electric field of multiple charge distributions
	\begin{itemize}
		\item You do not need to memorize any of the field equations in the book. If I want you to use an equation for the field of a charge distribution, I will either give it to you or ask you to derive it as a part of the problem.
		\item Example problems: P31, P53, P58
	\end{itemize}
\end{itemize}
\section*{Chapter 16}
\begin{itemize}
	\item Know the connection between potential difference $\Delta V$, change in potential energy $\Delta U$, and change in kinetic energy $\Delta K$.
	\begin{itemize}
		\item[$\ast$] $\Delta K = W = -\Delta U$
		\item[$\ast$] $\Delta U = q\Delta V$
		\item[$\ast$] $\Delta V = -\int\vec{E}\cdot \vec{dr}$
		\begin{itemize}
			\item[--] if $\vec{E}$ is uniform: $\Delta V = -\vec{E}\cdot\vec{\Delta r}$
		\end{itemize}
		\item[$\ast$] Example problems: P28, P37, P42
	\end{itemize}
	\item Know the meaning of the sign of potential difference (for a positive particle,  $-\Delta V\implies -\Delta U \implies + \Delta K$, particle gains kinetic energy and therefore speed.)
	\begin{itemize}
		\item Particles will move in the direction of \textit{decreasing} potential energy.
		\item Positive particles: move in the direction where $\Delta V$ is negative (move from high potential to low potential).
		\item Negative particles: opposite
		\item Example problems: P29, P30, P32
	\end{itemize}
	\item Given a function for electric potential: be able to calculate the electric field:
	\begin{eqnarray*}
	\vec{E}=-\left<\frac{dV}{dx},\frac{dV}{dy},\frac{dV}{dz}  \right>
	\end{eqnarray*}
\begin{itemize}
	\item Example problems: P33, P48
\end{itemize}
	\item Be familiar with the so-called ``potential at a single point'' $V$, which is simply useful shorthand the potential difference between that point and infinity
	\begin{itemize}
		\item Know the potential of a point charge at a point in space, $V=\frac{1}{4\pi\epsilon_0}\frac{q}{r}$
		\item Be able to use superposition to find the total potential of several point charges
		\item Be prepared to use this result to determine the electric field and potential energy of the charge distribution (note: since $\Delta U = q\Delta V, U=qV$)
		\item Example problems: P71, P74
	\end{itemize}
\end{itemize}
\end{document}