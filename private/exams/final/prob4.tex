\question Consider an ideal gas in a 2D ``flatland'' Universe. In this case, instead of the gas occupying a volume $V$, it occupies an area $A$. Instead of exerting a pressure $P=$ force/area with units of $N/m^2$, the gas exerts a ``linear pressure'' $\Pi$ = force/length with units of $N/m$. The gas still exchanges heat $SdT$ with its surroundings, but the work done on the gas is $-\Pi dA$ rather than $-PdV$. The infinitesimal change in energy of the gas is then:
\begin{equation*}
	dU=TdS-\Pi dA +\mu dN
\end{equation*}
For this 2D gas, the multiplicity is given by:
\begin{equation*}
	\Omega(U,A,N)=\frac{1}{N!}\left(\frac{A}{\hbar^2}\right)^N\frac{\pi^N}{N!}\left(2mU\right)^N
\end{equation*}
\begin{parts}
	\part[15] Derive the energy vs temperature relationship for this gas (in 3D, it is $U=\frac{3}{2}NkT$, what is it in 2D?)
	\vspace{5cm}
	\part[15] Derive the 2D ideal gas law (in 3D, it is $PV=NkT$, what is the 2D equivalent?)
\end{parts}