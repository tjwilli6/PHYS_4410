\documentclass{beamer}
\usetheme{default}
\hypersetup{colorlinks=true,urlcolor=blue,urlbordercolor=blue}
\title{The Quantum Distribution Functions}
%\author{TeXstudio Team}
\begin{document}
\begin{frame}[plain]
    \maketitle
\end{frame}
\begin{frame}{Expected number of particles in state $s$}
	\begin{align*}
		\left<n_s\right>&=\sum_{R}n_sP(n_s)\\
		\left<n_s\right>&=\frac{\sum_{R}n_se^{-\beta E_R}}{\sum_{R}e^{-\beta E_R}}
	\end{align*}
\end{frame}



\begin{frame}{Example: Harmonic Oscillator}
	3 particles distributed among 4 states
	\renewcommand{\arraystretch}{1.5}
	\centering
	\begin{tabular}{|c|c|c|c|c|c|}
		\hline
		R/s&n=0&n=1&n=2&n=3&$E_R$\\
		\hline
		1&3&0&0&0&$\frac{3}{2}\hbar\omega$\\
		\hline
		2&0&3&0&0&$\frac{9}{2}\hbar\omega$\\
		\hline
		3&0&0&3&0&$\frac{15}{2}\hbar\omega$\\
		\hline
		4&0&0&0&3&$\frac{21}{2}\hbar\omega$\\
		\hline
		5&2&1&0&0&$\frac{5}{2}\hbar\omega$\\
		\hline
		6&2&0&1&0&$\frac{7}{2}\hbar\omega$\\
		\hline
		7&2&0&0&1&$\frac{9}{2}\hbar\omega$\\
		\hline
		8&0&2&1&0&$\frac{11}{2}\hbar\omega$\\
		\hline
		9&0&2&0&1&$\frac{13}{2}\hbar\omega$\\
		\hline
		10&1&2&0&0&$\frac{7}{2}\hbar\omega$\\
		\hline
	\end{tabular}
\end{frame}



\begin{frame}{Example: Harmonic Oscillator}
	(continued...)\\
	\centering
	\renewcommand{\arraystretch}{1.5}
	\begin{tabular}{|c|c|c|c|c|c|}
		\hline
		R/s&n=0&n=1&n=2&n=3&$E_R$\\
		\hline
		11&0&0&2&1&$\frac{17}{2}\hbar\omega$\\
		\hline
		12&1&0&2&0&$\frac{11}{2}\hbar\omega$\\
		\hline
		13&0&1&2&0&$\frac{13}{2}\hbar\omega$\\
		\hline
		14&1&0&0&2&$\frac{15}{2}\hbar\omega$\\
		\hline
		15&0&1&0&2&$\frac{17}{2}\hbar\omega$\\
		\hline
		16&0&0&1&2&$\frac{19}{2}\hbar\omega$\\
		\hline
		17&1&1&1&0&$\frac{9}{2}\hbar\omega$\\
		\hline
		18&0&1&1&1&$\frac{15}{2}\hbar\omega$\\
		\hline
		19&1&0&1&1&$\frac{13}{2}\hbar\omega$\\
		\hline
		20&1&1&0&1&$\frac{11}{2}\hbar\omega$\\
		\hline
	\end{tabular}
\end{frame}





\begin{frame}{Example: Harmonic Oscillator}
	\begin{columns}
		\begin{column}{0.5\textwidth}
	\renewcommand{\arraystretch}{1.5}
	\setlength{\tabcolsep}{1pt}
	\begin{tabular}{|c|c|c|c|c|c|}
		\hline
		R/s&n=0&n=1&n=2&n=3&$E_R$\\
		\hline
		1&3&0&0&0&$\frac{3}{2}\hbar\omega$\\
		\hline
		2&0&3&0&0&$\frac{9}{2}\hbar\omega$\\
		\hline
		3&0&0&3&0&$\frac{15}{2}\hbar\omega$\\
		\hline
		4&0&0&0&3&$\frac{21}{2}\hbar\omega$\\
		\hline
		5&2&1&0&0&$\frac{5}{2}\hbar\omega$\\
		\hline
		6&2&0&1&0&$\frac{7}{2}\hbar\omega$\\
		\hline
		7&2&0&0&1&$\frac{9}{2}\hbar\omega$\\
		\hline
		8&0&2&1&0&$\frac{11}{2}\hbar\omega$\\
		\hline
		9&0&2&0&1&$\frac{13}{2}\hbar\omega$\\
		\hline
		10&1&2&0&0&$\frac{7}{2}\hbar\omega$\\
		\hline
	\end{tabular}
		\end{column}
	\begin{column}{0.5\textwidth}
	\renewcommand{\arraystretch}{1.5}
\setlength{\tabcolsep}{1pt}
\begin{tabular}{c}
$\sum_{R}n_2e^{-\beta E_R}=$\\
$0\cdot e^{-\frac{3}{2}\beta \hbar \omega}+$\\
$0\cdot e^{-\frac{9}{2}\beta \hbar \omega}+$\\
$3\cdot e^{-\frac{15}{2}\beta \hbar \omega}+$\\
$0\cdot e^{-\frac{21}{2}\beta \hbar \omega}+$\\
$0\cdot e^{-\frac{5}{2}\beta \hbar \omega}+$\\
$1\cdot e^{-\frac{7}{2}\beta \hbar \omega}+$\\
$0\cdot e^{-\frac{9}{2}\beta \hbar \omega}+$\\
$1\cdot e^{-\frac{11}{2}\beta \hbar \omega}+$\\
$0\cdot e^{-\frac{13}{2}\beta \hbar \omega}+$\\
$0\cdot e^{-\frac{7}{2}\beta \hbar \omega}+\cdots$
\end{tabular}
	\end{column}
	\end{columns}
\end{frame}



\begin{frame}{Expected number of particles in state $s$}
	\begin{align*}
		\left<n_s\right>&=\frac{\sum_{R}n_se^{-\beta E_R}}{\sum_{R}e^{-\beta E_R}}=\frac{\sum_{R}n_se^{-\beta\left(n_1\epsilon_1+n_2\epsilon_2+\cdots+n_s\epsilon_s+\cdots\right)}}{\sum_{R}e^{-\beta\left(n_1\epsilon_1+n_2\epsilon_2+\cdots+n_s\epsilon_s+\cdots\right)}}\\ \\
		&=\frac{\sum_{R}n_se^{\beta n_s\epsilon_s}e^{-\beta\left(n_1\epsilon_1+n_2\epsilon_2+\cdots+n_{s-1}\epsilon_{s-1}+\cdots\right)}}{\sum_{R}e^{-\beta n_s\epsilon_s}e^{-\beta\left(n_1\epsilon_1+n_2\epsilon_2+\cdots+n_{s-1}\epsilon_{s-1}+\cdots\right)}}
	\end{align*}
Here we factor terms depending on state $s$ from the sum
\end{frame}





\begin{frame}{Expected number of particles in state $s$}
	\begin{align*}
		\left<n_s\right>&=\frac{\sum_{R}n_se^{\beta n_s\epsilon_s}e^{-\beta\left(n_1\epsilon_1+n_2\epsilon_2+\cdots+n_{s-1}\epsilon_{s-1}+\cdots\right)}}{\sum_{R}e^{-\beta n_s\epsilon_s}e^{-\beta\left(n_1\epsilon_1+n_2\epsilon_2+\cdots+n_{s-1}\epsilon_{s-1}+\cdots\right)}}\\ \\
		&=\frac{\sum_{n_s}n_se^{-\beta n_s\epsilon_s}\sum_{R^{(s)}}e^{-\beta\left(n_1\epsilon_1+n_2\epsilon_2+\cdots+n_{s-1}\epsilon_{s-1}+\cdots\right)}}{\sum_{n_s}e^{-\beta n_s\epsilon_s}\sum_{R^{(s)}}e^{-\beta\left(n_1\epsilon_1+n_2\epsilon_2+\cdots+n_{s-1}\epsilon_{s-1}+\cdots\right)}}
	\end{align*}
	And we can sum over system $s$ separately
\end{frame}



\begin{frame}{Expected number of particles in state $s$}
	\begin{align*}
		\left<n_s\right>&=\frac{\sum_{R}n_se^{-\beta n_s\epsilon_s}e^{-\beta\left(n_1\epsilon_1+n_2\epsilon_2+\cdots+n_{s-1}\epsilon_{s-1}+\cdots\right)}}{\sum_{R}e^{-\beta n_s\epsilon_s}e^{-\beta\left(n_1\epsilon_1+n_2\epsilon_2+\cdots+n_{s-1}\epsilon_{s-1}+\cdots\right)}}\\ \\
		&=\frac{\sum_{n_s}n_se^{-\beta n_s\epsilon_s}\sum_{R^{(s)}}e^{-\beta\left(n_1\epsilon_1+n_2\epsilon_2+\cdots+n_{s-1}\epsilon_{s-1}+\cdots\right)}}{\sum_{n_s}e^{-\beta n_s\epsilon_s}\sum_{R^{(s)}}e^{-\beta\left(n_1\epsilon_1+n_2\epsilon_2+\cdots+n_{s-1}\epsilon_{s-1}+\cdots\right)}}
	\end{align*}
$\sum_{R^{(s)}}\rightarrow$sum over all microstates of the subsystem excluding state $s$ (subject to the constraint that $n_s+\sum_{i,i\neq s}n_i=N$)
\end{frame}


\begin{frame}{Example}
	If $N=3$, and there are 4 single particle states
	\footnotesize
	\begin{align*}
		\sum_{n_s}n_s&e^{-\beta n_s\epsilon_s}\sum_{R^{(s)}}e^{-\beta\left(n_1\epsilon_1+n_2\epsilon_2+\cdots+n_{s-1}\epsilon_{s-1}+\cdots\right)}=\\\\
		&0\cdot e^{-\beta\cdot 0\cdot \epsilon_2}\left(e^{-\beta\left(3\cdot\epsilon_1+0\cdot\epsilon_3+0\cdot\epsilon_4\right)}+e^{-\beta\left(0\cdot \epsilon_1+3\cdot\epsilon_3+0\cdot\epsilon_4\right)}+e^{-\beta\left(0\cdot\epsilon_1+0\cdot\epsilon_3+3\cdot\epsilon_4\right)}+\cdots\right)+\\ \\
		&1\cdot e^{-\beta\cdot 1\cdot\epsilon_2}\left(e^{-\beta\left(2\cdot\epsilon_1+0\cdot\epsilon_3+0\cdot\epsilon_4\right)}+e^{-\beta\left(0\cdot\epsilon_1+2\cdot\epsilon_3+0\cdot\epsilon_4\right)}+e^{-\beta\left(0\cdot\epsilon_1+0\cdot\epsilon_3+2\cdot\epsilon_4\right)}+\cdots\right)+\\ \\
		&2\cdot e^{-\beta\cdot 2\cdot\epsilon_2}\left(e^{-\beta\left(1\cdot\epsilon_1+0\cdot\epsilon_3+0\cdot\epsilon_4\right)}+e^{-\beta\left(0\cdot\epsilon_1+1\cdot\epsilon_3+0\cdot\epsilon_4\right)}+e^{-\beta\left(0\cdot\epsilon_1+0\cdot\epsilon_3+1\cdot\epsilon_4\right)}\right)+\\ \\
		&3\cdot e^{-\beta\cdot 3\epsilon_2}\left(e^{-\beta\left(0\cdot\epsilon_1+0\cdot\epsilon_3+0\cdot\epsilon_4\right)}\right)
	\end{align*}
\end{frame}



\begin{frame}{Expected number of particles in state $s$}
	Note that 
	\begin{equation*}
		\sum_{R}e^{-\beta\left(n_1\epsilon_1+n_2\epsilon_2+\cdots+n_s\epsilon_s+\cdots\right)}
	\end{equation*}
	Is just the partition function of the entire system.
\end{frame}



\begin{frame}{Expected number of particles in state $s$}
	Note that 
	\begin{equation*}
			\sum_{R}e^{-\beta\left(n_1\epsilon_1+n_2\epsilon_2+\cdots+n_s\epsilon_s+\cdots\right)}
	\end{equation*}
	So we interpret the quantity
	\begin{equation*}
		\sum_{R^{(s)}}e^{-\beta\left(n_1\epsilon_1+n_2\epsilon_2+\cdots+n_{s-1}\epsilon_{s-1}+\cdots\right)}
	\end{equation*}
As the partition function of the subsystem consisting of:
\begin{itemize}
	\item All system states \textit{except} $s$
	\item All $N$ system particles \textit{except} $n_s$
\end{itemize}
\end{frame}


\begin{frame}{Expected number of particles in state $s$}
	And we introduce the following definition:
	\begin{equation*}
		Z_s(N-n_s)\equiv\sum_{R^{(s)}}e^{-\beta\left(n_1\epsilon_1+n_2\epsilon_2+\cdots+n_{s-1}\epsilon_{s-1}+\cdots\right)}
	\end{equation*}
\end{frame}



\begin{frame}{Expected number of particles in state $s$}
	\begin{align*}
		\left<n_s\right>=\frac{\sum_{n_s}n_se^{-\beta n_s\epsilon_s}Z_s(N-n_s)}{\sum_{n_s}e^{-\beta n_s\epsilon_s}Z_s(N-n_s)}
	\end{align*}
\end{frame}




\begin{frame}{$\left<n_s\right>$ for fermions}
	\begin{align*}
		\left<n_s\right>=\frac{\sum_{n_s}n_se^{-\beta n_s\epsilon_s}Z_s(N-n_s)}{\sum_{n_s}e^{-\beta n_s\epsilon_s}Z_s(N-n_s)}
	\end{align*}
For \textbf{fermions} $\sum_{n_s}$ is simple:
\begin{itemize}
	\item $n_s$ can only be $0$ or $1$
\end{itemize}
\end{frame}




\begin{frame}{$\left<n_s\right>$ for fermions}
	\begin{align*}
		\left<n_s\right>&=\frac{\sum_{n_s}n_se^{-\beta n_s\epsilon_s}Z_s(N-n_s)}{\sum_{n_s}e^{-\beta n_s\epsilon_s}Z_s(N-n_s)}\\\\
		&=\frac{0\cdot e^{-\beta\cdot 0\cdot\epsilon_s}Z_s(N)+1\cdot e^{-\beta\cdot1\cdot\epsilon_s}Z_s(N-1)}{e^{-\beta\cdot0\cdot\epsilon_s}Z_s(N)+e^{-\beta\cdot1\cdot\epsilon_s}Z_s(N-1)}\\\\
		&=\frac{e^{-\beta\epsilon_s}Z_s(N-1)}{Z_s(N)+e^{-\beta\epsilon_s}Z_s(N-1)}
	\end{align*}
\end{frame}




\begin{frame}{$\left<n_s\right>$ for fermions}
	\begin{align*}
		\left<n_s\right>
		&=\frac{e^{-\beta\epsilon_s}Z_s(N-1)}{Z_s(N)+e^{-\beta\epsilon_s}Z_s(N-1)}\\\\
		&=\frac{1}{\frac{Z_s(N)}{Z_s(N-1)}e^{-\beta\epsilon_s}+1}
	\end{align*}
\end{frame}




\begin{frame}{$\left<n_s\right>$ for fermions}
	\begin{align*}
		\left<n_s\right>
		&=\frac{1}{\frac{Z_s(N)}{Z_s(N-1)}e^{\beta\epsilon_s}+1}
	\end{align*}
\end{frame}




\begin{frame}{$\left<n_s\right>$ for fermions}
	Now we use a Taylor expansion to relate $Z_s(N)$ to $Z_s(N-1)$ (since $N>>1$)
	\begin{align*}
		\ln{Z_s(N-1)}\approx \ln{Z_s(N)}-1\cdot\frac{\partial}{\partial N}\ln{Z_s(N)}
	\end{align*}

\end{frame}


\begin{frame}{$\left<n_s\right>$ for fermions}
	Now we use a Taylor expansion to relate $Z_s(N)$ to $Z_s(N-1)$ (since $N>>1$)
	\begin{align*}
		\ln{Z_s(N-1)}\approx \ln{Z_s(N)}-1\cdot\frac{\partial}{\partial N}\ln{Z_s(N)}
	\end{align*}
	Recall that:
	\begin{equation*}
		F=-kT\ln{Z_s(N)}
	\end{equation*}
	
\end{frame}



\begin{frame}{$\left<n_s\right>$ for fermions}
	Now we use a Taylor expansion to relate $Z_s(N)$ to $Z_s(N-1)$ (since $N>>1$)
	\begin{align*}
		\ln{Z_s(N-1)}\approx \ln{Z_s(N)}-1\cdot\frac{\partial}{\partial N}\ln{Z_s(N)}
	\end{align*}
	Recall that:
	\begin{equation*}
		F=-kT\ln{Z_s(N)}
	\end{equation*}
	So:
	\begin{equation*}
		\frac{\partial}{\partial N}\ln{Z_s(N)}=-\frac{1}{kT}\frac{\partial F}{\partial N}=-\beta\frac{\partial F}{\partial N}
	\end{equation*}
\end{frame}




\begin{frame}{$\left<n_s\right>$ for fermions}
	Now we use a Taylor expansion to relate $Z_s(N)$ to $Z_s(N-1)$ (since $N>>1$)
	\begin{align*}
		\ln{Z_s(N-1)}\approx \ln{Z_s(N)}-1\cdot\frac{\partial}{\partial N}\ln{Z_s(N)}
	\end{align*}
	Recall that:
	\begin{equation*}
		F=-kT\ln{Z_s(N)}
	\end{equation*}
	So:
	\begin{equation*}
		\frac{\partial}{\partial N}\ln{Z_s(N)}=-\frac{1}{kT}\frac{\partial F}{\partial N}=-\beta\frac{\partial F}{\partial N}
	\end{equation*}
And: 
\begin{equation*}
	\frac{\partial F}{\partial N}=\mu
\end{equation*}
\end{frame}



\begin{frame}{$\left<n_s\right>$ for fermions}
	So:
	\begin{equation*}
		\frac{\partial}{\partial N}\ln{Z_s(N)}=-\beta \mu
	\end{equation*}
And:
	\begin{align*}
		\ln{Z_s(N-1)}&\approx \ln{Z_s(N)}-1\cdot\frac{\partial}{\partial N}\ln{Z_s(N)}\\\\
		&=  \ln{Z_s(N)}+\mu\beta
	\end{align*}
\end{frame}




\begin{frame}{$\left<n_s\right>$ for fermions}
	\begin{align*}
		\ln{Z_s(N-1)}&\approx \ln{Z_s(N)}+\mu\beta\\\\
	\end{align*}
So:
	\begin{align*}
		Z_s(N-1)&\approx Z_s(N)e^{\mu\beta}
	\end{align*}
And:
	\begin{align*}
		\frac{Z_s(N)}{Z_s(N-1)}\approx e^{-\mu\beta}
	\end{align*}
\end{frame}




\begin{frame}{$\left<n_s\right>$ for fermions}
	\begin{align*}
		\frac{Z_s(N)}{Z_s(N-1)}\approx e^{-\mu\beta}
	\end{align*}
And:
	\begin{align*}
	\left<n_s\right>
	&=\frac{1}{\frac{Z_s(N)}{Z_s(N-1)}e^{\beta\epsilon_s}+1}
\end{align*}
So:
\begin{align*}
	\left<n_s\right>=\frac{1}{e^{\beta\left(\epsilon_s-\mu\right)}+1}
\end{align*}
\end{frame}



\begin{frame}{$\left<n_s\right>$ for fermions}
	\begin{equation*}
		\boxed{	\left<n_s\right>=\frac{1}{e^{\beta\left(\epsilon_s-\mu\right)}+1}}
	\end{equation*}

\end{frame}



\begin{frame}{$\left<n_s\right>$ for \textit{bosons}}
	\begin{align*}
	\left<n_s\right>=\frac{\sum_{n_s}n_se^{-\beta n_s\epsilon_s}Z_s(N-n_s)}{\sum_{n_s}e^{-\beta n_s\epsilon_s}Z_s(N-n_s)}
\end{align*}
\end{frame}

\begin{frame}{$\left<n_s\right>$ for \textit{bosons}}
	\begin{align*}
		\left<n_s\right>=\frac{\sum_{n_s}n_se^{-\beta n_s\epsilon_s}Z_s(N-n_s)}{\sum_{n_s}e^{-\beta n_s\epsilon_s}Z_s(N-n_s)}
	\end{align*}
	For \textbf{bosons}, $\sum_{n_s}$ ranges from $n_s=0$ up to $n_s=N$
\end{frame}




\begin{frame}{$\left<n_s\right>$ for \textit{bosons}}
	\tiny
	\begin{align*}
		\left<n_s\right>&=\frac{\sum_{n_s}n_se^{-\beta n_s\epsilon_s}Z_s(N-n_s)}{\sum_{n_s}e^{-\beta n_s\epsilon_s}Z_s(N-n_s)}\\\\
		&=\frac{0\cdot e^{\beta\cdot 0\cdot \epsilon_s}\cdot Z_s(N)+1\cdot e^{-\beta\cdot\epsilon_s}\cdot Z_s(N-1)+2\cdot e^{-\beta\cdot2\cdot\epsilon_s}\cdot Z_s(N-2)+3\cdot e^{-\beta\cdot3\cdot\epsilon_s}\cdot Z_s(N-3)+\cdots}{ e^{\beta\cdot 0\cdot \epsilon_s}\cdot Z_s(N)+ e^{-\beta\cdot\epsilon_s}\cdot Z_s(N-1)+ e^{-\beta\cdot2\cdot\epsilon_s}\cdot Z_s(N-2)+ e^{-\beta\cdot3\cdot\epsilon_s}\cdot Z_s(N-3)+\cdots}
	\end{align*}
\end{frame}



\begin{frame}{$\left<n_s\right>$ for bosons}
	Again using a Taylor expansion:
	\begin{align*}
		\ln{Z_s(N-\Delta N)}\approx \ln{Z_s(N)}-\Delta N\cdot\frac{\partial}{\partial N}\ln{Z_s(N)}
	\end{align*}
	
\end{frame}


\begin{frame}{$\left<n_s\right>$ for bosons}
	Again using a Taylor expansion:
	\begin{align*}
		\ln{Z_s(N-\Delta N)}\approx \ln{Z_s(N)}-\Delta N\cdot\frac{\partial}{\partial N}\ln{Z_s(N)}
	\end{align*}
	And recalling that:
	\begin{equation*}
		\frac{\partial}{\partial N}\ln{Z_s(N)}=\beta\frac{\partial F}{\partial N}=-\beta \mu
	\end{equation*}
\end{frame}


\begin{frame}{$\left<n_s\right>$ for bosons}
	Again using a Taylor expansion:
	\begin{align*}
		\ln{Z_s(N-\Delta N)}\approx \ln{Z_s(N)}-\Delta N\cdot\frac{\partial}{\partial N}\ln{Z_s(N)}
	\end{align*}
	And recalling that:
	\begin{equation*}
		\frac{\partial}{\partial N}\ln{Z_s(N)}=\beta\frac{\partial F}{\partial N}=-\beta \mu
	\end{equation*}

We find:
\begin{equation*}
	\frac{Z_s(N)}{Z_s(N-\Delta N)}\approx e^{-\Delta N\mu \beta}
\end{equation*}
\end{frame}

\begin{frame}{$\left<n_s\right>$ for bosons}
	Using this fact: we can rewrite $\left<n_s\right>$:
	\tiny
\begin{align*}
	\left<n_s\right>&=\frac{0\cdot e^{\beta\cdot 0\cdot \epsilon_s}\cdot Z_s(N)+1\cdot e^{-\beta\cdot\epsilon_s}\cdot Z_s(N-1)+2\cdot e^{-\beta\cdot2\cdot\epsilon_s}\cdot Z_s(N-2)+3\cdot e^{-\beta\cdot3\cdot\epsilon_s}\cdot Z_s(N-3)+\cdots}{ e^{\beta\cdot 0\cdot \epsilon_s}\cdot Z_s(N)+ e^{-\beta\cdot\epsilon_s}\cdot Z_s(N-1)+ e^{-\beta\cdot2\cdot\epsilon_s}\cdot Z_s(N-2)+ e^{-\beta\cdot3\cdot\epsilon_s}\cdot Z_s(N-3)+\cdots}\\ \\
	&=\frac{Z_s(N)\left(0+\frac{Z_s(N-1)}{Z_s(N)}e^{-\beta \epsilon_s}+2\frac{Z_s(N-2)}{Z_s(N)}e^{-2\beta\epsilon_s}+\cdots\right)}{Z_s(N)\left(0+\frac{Z_s(N-1)}{Z_s(N)}e^{-\beta \epsilon_s}+\frac{Z_s(N-2)}{Z_s(N)}e^{-2\beta\epsilon_s}+\cdots\right)}
\end{align*}
	
\end{frame}




\begin{frame}{$\left<n_s\right>$ for bosons}
	\begin{align*}
		\left<n_s\right>
		&=\frac{Z_s(N)\left(0+\frac{Z_s(N-1)}{Z_s(N)}e^{-\beta \epsilon_s}+2\frac{Z_s(N-2)}{Z_s(N)}e^{-2\beta\epsilon_s}+\cdots\right)}{Z_s(N)\left(0+\frac{Z_s(N-1)}{Z_s(N)}e^{-\beta \epsilon_s}+\frac{Z_s(N-2)}{Z_s(N)}e^{-2\beta\epsilon_s}+\cdots\right)}
	\end{align*}

And:
\begin{align*}
	\frac{Z_s(N-\Delta N)}{Z_s(N)}\approx e^{\Delta N\beta \mu}
\end{align*}
	
\end{frame}





\begin{frame}{$\left<n_s\right>$ for bosons}
	So:
	\begin{align*}
		\left<n_s\right>
		&=\frac{Z_s(N)\left(0+e^{\beta \mu}e^{-\beta \epsilon_s}+2e^{2\beta\mu}e^{-2\beta\epsilon_s}+\cdots\right)}{Z_s(N)\left(0+e^{\beta\mu}e^{-\beta \epsilon_s}+e^{2\beta\mu}e^{-2\beta\epsilon_s}+\cdots\right)}\\\\
		&=\frac{e^{-\beta\left(\epsilon_s-\mu\right)}+2e^{-2\beta\left(\epsilon_s-\mu\right)}+3e^{-3\beta\left(\epsilon_s-\mu\right)}+\cdots}{1+e^{-\beta\left(\epsilon_s-\mu\right)}+e^{-2\beta\left(\epsilon_s-\mu\right)}+e^{-3\beta\left(\epsilon_s-\mu\right)}+\cdots}\\\\
		&=\frac{\sum_{n_s}n_se^{-\beta n_s\left(\epsilon_s-\mu\right)}}{\sum_{n_s}e^{-\beta n_s\left(\epsilon_s-\mu\right)}}
	\end{align*}
	
	
\end{frame}





\begin{frame}{$\left<n_s\right>$ for bosons}
	So:
	\begin{align*}
		\left<n_s\right>
		&=\frac{\sum_{n_s}n_se^{-\beta n_s\left(\epsilon_s-\mu\right)}}{\sum_{n_s}e^{-\beta n_s\left(\epsilon_s-\mu\right)}}
	\end{align*}
	
	
\end{frame}



\begin{frame}{$\left<n_s\right>$ for bosons}
	So:
	\begin{align*}
		\left<n_s\right>
		&=\frac{\sum_{n_s}n_se^{-\beta n_s\left(\epsilon_s-\mu\right)}}{\sum_{n_s}e^{-\beta n_s\left(\epsilon_s-\mu\right)}}
	\end{align*}
	Which we write as:
	\begin{align*}
		\left<n_s\right>&=\frac{\sum_{n_s}-\frac{1}{\beta}\frac{\partial}{\partial \epsilon_s}e^{-\beta n_s\left(\epsilon_s-\mu\right)}}{\sum_{n_s}e^{-\beta n_s\left(\epsilon_s-\mu\right)}}\\\\
		&=\frac{-\frac{1}{\beta}\frac{\partial}{\partial \epsilon_s}\sum_{n_s}e^{-\beta n_s\left(\epsilon_s-\mu\right)}}{\sum_{n_s}e^{-\beta n_s\left(\epsilon_s-\mu\right)}}\\\\
		&=-\frac{1}{\beta}\frac{\partial}{\partial \epsilon_s}\ln{\left(\sum_{n_s}e^{-\beta n_s\left(\epsilon_s-\mu\right)}\right)}
	\end{align*}
	
\end{frame}




\begin{frame}{$\left<n_s\right>$ for bosons}
	\begin{align*}
		\left<n_s\right>&=-\frac{1}{\beta}\frac{\partial}{\partial \epsilon_s}\ln{\left(\sum_{n_s}e^{-\beta n_s\left(\epsilon_s-\mu\right)}\right)}
	\end{align*}
	$\sum_{n_s}e^{-\beta n_s\left(\epsilon_s-\mu\right)}$ is a \textbf{geometric series}:
	\begin{equation*}
		S_N=\sum_{n_s}^{N}r^{n_s}
	\end{equation*}
Where $r=e^{-\beta\left(\epsilon_s-\mu\right)}$
\end{frame}



\begin{frame}{$\left<n_s\right>$ for bosons}
	We showed how to solve this series in class:
	\begin{align*}
		S_N=\sum_{n_s}^{N}r^{n_s}=\frac{1-r^{N+1}}{1-r}
	\end{align*}

If $|r|$=$\left|e^{-\beta\left(\epsilon_s-\mu\right)}\right|<1$, and $N>>1$:
\begin{equation*}
\sum_{n_s}e^{-\beta n_s\left(\epsilon_s-\mu\right)}=\frac{1}{1-e^{-\beta\left(\epsilon_s-\mu\right)}}
\end{equation*}
\end{frame}

	
\begin{frame}{$\left<n_s\right>$ for bosons}
	Finally:
	\begin{align*}
		\left<n_s\right>&=-\frac{1}{\beta}\frac{\partial}{\partial \epsilon_s}\ln{\left(\sum_{n_s}e^{-\beta n_s\left(\epsilon_s-\mu\right)}\right)}\\\\
		&=-\frac{1}{\beta}\frac{\partial}{\partial \epsilon_s}\ln{\left(\frac{1}{1-e^{-\beta\left(\epsilon_s-\mu\right)}}\right)}\\\\
		&=\frac{1}{e^{\beta\left(\epsilon_s-\mu\right)}-1}
	\end{align*}
\end{frame}



\begin{frame}{$\left<n_s\right>$ for bosons}
	\begin{equation*}
		\boxed{	\left<n_s\right>=\frac{1}{e^{\beta\left(\epsilon_s-\mu\right)}-1}}
	\end{equation*}
\end{frame}



\begin{frame}{The quantum \textit{distribution functions}}
	\textbf{Fermions} (\textit{Fermi-Dirac statistics})
	\begin{equation*}
		\left<n_s\right>=\frac{1}{e^{\beta\left(\epsilon_s-\mu\right)}+1}
	\end{equation*}


	\textbf{Bosons} (\textit{Bose-Einstein statistics})
\begin{equation*}
	\left<n_s\right>=\frac{1}{e^{\beta\left(\epsilon_s-\mu\right)}-1}
\end{equation*}
\end{frame}
\end{document}
